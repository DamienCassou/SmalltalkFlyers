\newcommand{\stSmalltalkSubtitle}
{un langage de programmation\\
  purement \textbf{orienté objet}\\
  et un environnement \textbf{dynamique}}

\newcommand{\stSmalltalkConceptsTerm}{Concepts importants de Smalltalk}
\newcommand{\stSmalltalkConceptsDefinition}
{Smalltalk est un langage \emph{orienté objet}, à \emph{typage
    dynamique} dont la syntaxe est minimale et peut s'apprendre en
  \emph{quinze minutes}.

  Sa principale force vient du fait qu'il soit \emph{très cohérent} :
\begin{itemize}
\item tout est objet : les classes, les méthodes, les nombres, etc.
\item très peu de règles et aucune exception.
\end{itemize}

Smalltalk fonctionne sur le principe d'une \emph{machine virtuelle}.
Le développement se fait dans une \emph{image mémoire} dans laquelle
se trouve l'ensemble des objets du système avec lesquels il est
possible d'interagir.}

\newcommand{\stSmalltalkSyntaxTerm}{Syntaxe Smalltalk}
\newcommand{\stReservedWordsTerm}{Mots réservés}
\newcommand{\stReservedCaractersTerm}{Caractères réservés}

\newcommand{\stMessageSendingTerm}{Envoi de messages}
\newcommand{\stMessageSendingDefinition}
{L'appel de méthode se fait par envoi de message. Le message se
  construit sur la base du langage naturel avec un sujet, un verbe et
  des compléments. Tout envoi de message retourne un objet. Tous les
  messages sont envoyés à un objet que l'on appelle le receveur du
  message. Il existe trois types de messages : unaire, binaire et à
  mots-clés.}

\newcommand{\stUnaryMessagesTerm}{Messages unaires.}
\newcommand{\stUnaryMessagesDefinition}
{Un message unaire n'a pas d'argument.

\begin{displaycode}
array := Array new.

array size.
\end{displaycode}


Le premier exemple crée et retourne une nouvelle instance de la classe
Array en lui envoyant le message \code{new}. Le deuxième exemple
demande la taille du tableau ce qui retourne \code{0}.}

\newcommand{\stBinaryMessagesTerm}{Messages binaires.}
\newcommand{\stBinaryMessagesDefinition}
{Un message binaire ne prend qu'un argument, se nomme par un symbole et
est souvent utilisé pour des opérations arithmétiques.

\begin{displaycode}
3 + 4.

'Hello', ' World'.
\end{displaycode}

Le message \code{+} est envoyé à l'objet \code{3} avec comme paramètre \code{4}. Dans le
second cas, le message \code{,} est envoyé à la chaîne de caractères
\code{'Hello'} avec \code{' World'} en paramètre.
}

\newcommand{\stKeywordMessagesTerm}{Messages à mots-clés.}
\newcommand{\stKeywordMessagesDefinition}
{
Un message à mots-clés peut prendre un ou plusieurs arguments. Les
arguments sont insérés entre chaque mot-clé, après les deux-points.

\begin{displaycode}
'Smalltalk' allButFirst: 5.

3 to: 10 by: 2.
\end{displaycode}

Le premier exemple appelle la méthode \code{allButFirst:} sur une chaîne de
caractères et passe l'argument \code{5}. La méthode retourne la chaîne de
caractères \code{'talk'}. Le deuxième exemple retourne une collection
contenant les éléments \code{3}, \code{5}, \code{7} et \code{9}.
}

\newcommand{\stDevelopmentEnvironmentTerm}{Environnement de développement}
\newcommand{\stDevelopmentEnvironmentDefinition}
{
La plupart des implémentations de Smalltalk sont fournies avec un
environnement de développement intégré qui permet de naviguer dans le
code et d'interagir avec les objets. De nombreux outils sont
disponibles, tous implémentés en Smalltalk grâce à une API de
réflexion :

\begin{itemize}
\item un navigateur de classes et de méthodes ;
\item des outils de refactorisation ;
\item un inspecteur d'objets ;
\item un débogueur ;
\item etc.
\end{itemize}

L'environnement permet d'évaluer du code par un simple raccourci
clavier et de voir immédiatement le résultat.
}

\newcommand{\stImplementationTerm}{Implémentations}
\newcommand{\stImplementationDefinition}
{
Il existe différentes implémentations de Smalltalk :
\begin{description}
\item[Squeak] : implémentation libre, gratuite et multi-plateforme. Activement
  développée par une communauté internationale.
\item[Visual Works] : implémentation propriétaire et multi-plateforme,
  disponible gratuitement pour une utilisation personnelle.
\item[Gemstone] : implémentation propriétaire qui intègre une base de
  données objet à hautes performances.
\item[Et d'autres] : GNU Smalltalk, Smalltalk/X, SyX, VA Smalltalk, Dolphin\dots
\end{description}
}

\newcommand{\stSqueakCodeBrowserTerm}{Le navigateur de code de Squeak}

\newcommand{\stApplicationsTerm}{Applications}
\newcommand{\stApplicationsDefinition}
{
\begin{description}
\item[Éducation] : EToys (Squeak), SqueakBot, BotsInc, Scratch\dots
\item[Multimédia] : Sophie, OpenCroquet, Plopp\dots
\item[Développement web] : Seaside, Aida, Komanche, Swazoo\dots
\item[Gestion de la persistance] : base de données objets (Magma, GemStone), relationnelles (MySQL, PostgreSQL), mapping objet relationnel (Glorp).
\end{description}
}

\newcommand{\stPloppDrawingSessionTerm}{Une session de dessin dans Plopp}

\newcommand{\stImageTerm}{Image mémoire}
\newcommand{\stImageDefinition}
{L'environnement Smalltalk contient un entrepôt persistant d'objets,
  l'image. Elle contient le code de l'application (les classes et les
  méthodes), les objets qui constituent l'état de l'application et
  peut même contenir l'environnement de programmation pour inspecter
  et déboguer le programme pendant qu'il s'exécute.}

\newcommand{\stVMTerm}{Machine virtuelle}
\newcommand{\stVMDefinition}
{Une machine virtuelle est un programme qui est capable d'exécuter
  d'autres programmes. Cela permet de faciliter la portabilité des
  applications que l'on développe.}

\newcommand{\stReflexionTerm}{Réflexion}
\newcommand{\stReflexionDefinition}
{On dit qu'un langage est réflexif s'il dispose de mécanismes
  permettant d'inspecter et de modifier du code pendant l'exécution
  d'un programme.}

\newcommand{\stDynamicTypingTerm}{Typage dynamique}
\newcommand{\stDynamicTypingDefinition}
{Certains langages forcent le développeur à indiquer le type de chaque
  variable (entier, chaîne de caractères\dots). On appelle cela le
  typage statique. En typage dynamique, le développeur ne contraint
  pas ses variables à un type particulier.}

\newcommand{\stBooksTerm}{Ouvrages}
\newcommand{\stBooksDefinition}
{
\begin{itemize}
\item Nombreux livres téléchargeables gratuitement\\
  \url{http://stephane.ducasse.free.fr/Books.html}

\item Smalltalk en général
  \begin{itemize}
  \item \emph{Smalltalk with Style}\\
		(Edward Klimas, Suzanne Skublics et
    David A. Thomas, gratuit)
  \item \emph{Smalltalk by Example: the Developer's Guide} --
		(Alec Sharp, gratuit)
  \end{itemize}

\item Squeak en particulier
  \begin{itemize}
  \item \emph{Squeak by Example} -- (2007, libre, traduction en cours)
  \item \emph{Squeak} -- (Xavier Briffault et Stéphane Ducasse, en français)
  \item \emph{Powerful Ideas in the Classroom}\\
		(BJ Allen-Conn et Kim Rose, en anglais et français)
  \end{itemize}
\end{itemize}
}

\newcommand{\stSmalltalkActionsTerm}{Manifestations}
\newcommand{\stSmalltalkActionsDefinition}
{
\begin{itemize}
\item Smalltalk Party : Journée francophone annuelle, organisée à
  Paris par les Smalltalkiens français.\\
  \url{http://community.ofset.org/wiki/SmalltalkParty}
\item Conférence du groupe européen des utilisateurs de Smalltalk
  (ESUG). Elle réunit chaque année, depuis 1993, les Smalltalkiens
  industriels et académiques dans un pays d'Europe.\\
  \url{http://www.esug.org/conferences}
\item Conférence annuelle, organisée en Amérique du nord par le STIC (\url{http://www.stic.st}),
  association qui réuni les grands acteurs industriels et éditeurs
  de Smalltalk.\\ \url{http://www.smalltalksolutions.com/}
\end{itemize}
}

\newcommand{\stInternetWebsitesDefinition}
{
\begin{itemize}
\item Site officiel en anglais :\\ \url{http://www.squeak.org}
\item Wiki francophone :\\ \url{http://community.ofset.org/wiki/Squeak}
\end{itemize}
}

\newcommand{\stNilDefinition}{objet indéfini (valeur par défaut des
  variables).}
\newcommand{\stTrueAndFalseDefinition}{objets booléens.}
\newcommand{\stSelfDefinition}{l'objet courant.}
\newcommand{\stSuperDefinition}{l'objet courant dans le contexte de la
  super classe.}
\newcommand{\stThisContextDefinition}{partie de la \emph{pile
    d'exécution} correspondant à la méthode courante.}
\newcommand{\stAssignmentOperatorDefinition}{affectation.}
\newcommand{\stReturnOperatorDefinition}{retour du résultat d'une méthode.}
\newcommand{\stTempsDeclarationOperatorDefinition}{déclaration de trois variables temporaires.}
\newcommand{\stDollarOperatorForCharacterADefinition}{le caractère \code{a}.}
\newcommand{\stLiteralArrayDefinition}{tableau contenant deux littéraux \code{\#abc} et \code{123}.}
\newcommand{\stDotOperatorDefinition}{termine toute expression.}
\newcommand{\stSemiColonOperatorDefinition}{cascade de messages.}
\newcommand{\stBlockOperatorDefinition}{bloc de code (c'est un objet !).}

\newcommand{\stCommentTerm}{commentaire}
\newcommand{\stStringTerm}{chaîne de caractères}

\newcommand{\stGlossaryTerm}{Glossaire}

\newcommand{\stAndTerm}{et}
\newcommand{\stOrTerm}{or}
%%% Local Variables:
%%% coding: utf-8-unix
%%% mode: latex
%%% TeX-master: flyer
%%% TeX-PDF-mode: t
%%% ispell-local-dictionary: "francais"
%%% End:
