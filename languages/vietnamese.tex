\newcommand{\stSmalltalkSubtitle}
{ngôn ngữ lập trình\\
  thuần \textbf{hướng đối tượng}\\
  và môi trường \textbf{động}}

\newcommand{\stSmalltalkConceptsTerm}{Những khái niệm quan trọng của Smalltalk}
\newcommand{\stSmalltalkConceptsDefinition} {Smalltalk là một ngôn ngữ
  \emph{hướng đối tượng} và \emph{định kiểu động}, với cú pháp đơn giản, có thể học được trong \emph{mười lăm phút}.
Lợi thế của chính của Smalltalk là nhờ vào sự \emph{nhất quán}:

\begin{itemize}
\item tất cả các thứ đều là đối tượng: classes (lớp), methods (phương thức), số, ...
\item số lượng quy tắc rất ít, và không có ngoại lệ!
\end{itemize}

Smalltalk chạy trên một \emph{máy ảo}.  Quá trình phát triển chương
trình thực hiện trên một \emph{image (ảnh)} nơi mà tất cả các đối
tượng được lưu trữ và sửa đổi.}

\newcommand{\stSmalltalkSyntaxTerm}{Cú pháp Smalltalk}
\newcommand{\stReservedWordsTerm}{Các từ khóa}
\newcommand{\stReservedCaractersTerm}{Các ký tự khóa}

\newcommand{\stMessageSendingTerm}{Gửi message (thông điệp)}
\newcommand{\stMessageSendingDefinition} {Một method được gọi bằng
  cách gửi một message (dựa trên ngôn ngữ tự nhiên) đến một đối tượng,
  message sẽ trả về một đối tượng khác.  Có ba loại message: unary,
  binary, và keyword.}

\newcommand{\stUnaryMessagesTerm}{Unary message (một ngôi).}
\newcommand{\stUnaryMessagesDefinition}
{Unary message là message không có đối số (argument).

\begin{displaycode}
array := Array new.

array size.
\end{displaycode}

Ví dụ đầu tiên này tạo và trả về một instance của class Array, bằng cách
gửi nó message \code{new}. Ví dụ thứ hai yêu cầu kích thước của mảng
này, kết quả là \code{0}.}

\newcommand{\stBinaryMessagesTerm}{Binary message (hai ngôi).}
\newcommand{\stBinaryMessagesDefinition}{Binary message chỉ có một đối
số, được đặt tên bằng một ký hiệu và thường được dùng cho
các biểu thức toán học.

\begin{displaycode}
3 + 4.

'Hello', ' World'.
\end{displaycode}

Message \code{+} được gửi đến đối tượng \code{3} với \code{4} là tham
số. Trong trường hợp thứ hai, message \code{,} được gửi đến xâu
\code{'Hello'} với \code{' World'} là tham số.}

\newcommand{\stKeywordMessagesTerm}{Keyword message (từ khóa).}
\newcommand{\stKeywordMessagesDefinition}
{
  Keyword message có thể có một hay nhiều đối số. Các đối số được chèn vào giữa mỗi từ khóa, sau mỗi dấu hai chấm.

\begin{displaycode}
'Smalltalk' allButFirst: 5.

3 to: 10 by: 2.
\end{displaycode}

Ví dụ đầu gọi method \code{allButFirst:} trên một xâu và
truyền đối số \code{5}. Trả về xâu \code{'talk'}. Ví dụ thứ
hai trả về một tập hợp chứa các phần tử là \code{3}, \code{5},
\code{7} and \code{9}.  }

\newcommand{\stDevelopmentEnvironmentTerm}{Môi trường phát triển}
\newcommand{\stDevelopmentEnvironmentDefinition} {Hầu hết mọi bản
  Smalltalk được cung cấp với một môi trường phát triển tích hợp
  (IDE), cho phép duyệt mã nguồn và tác động đến các đối tượng. Nhờ
  vào tính năng API reflection mà chúng ta có nhiều công cụ trong
  Smalltalk như:

\begin{itemize}
\item trình duyệt class và method (class and method browser);
\item các công cụ cải tiến mã nguồn (refactoring tools);
\item trình kiểm tra đối tượng (object inspectors);
\item trình bắt lỗi (debugger);
\item công cụ quản lý phiên bản (release management and version control tools);
\item và còn nhiều nữa!
\end{itemize}

Mã nguồn có thể được kiểm tra và thi hành trực tiếp trong image,
bằng cách dùng các tổ hợp phím đơn giản hoặc menu.}

\newcommand{\stImplementationTerm}{Các bản thi hành Smalltalk}
\newcommand{\stImplementationDefinition}
{
Có nhiều bản thi hành Smalltalk hiện tại:
\begin{description}
\item[Squeak \& Pharo:] tự do, nguồn mở, đa nền tảng. Được phát triển
  liên tục.
\item[VisualWorks:] chủ quyền, đa nền tảng, miễn phí cho ứng dụng
  phi thương mại.
\item[Gemstone:] chủ quyền, có object database hiệu suất cao.
\item[Và các bản khác:] GNU Smalltalk, Smalltalk/X, SyX, VA Smalltalk,
  Dolphin\dots
\end{description}
}

\newcommand{\stSqueakCodeBrowserTerm}{Trình duyệt mã Pharo}

\newcommand{\stApplicationsTerm}{Ứng dụng}
\newcommand{\stApplicationsDefinition} { Ngay từ khi ra đời vào những
  năm 1980, Smalltalk đã được sử dụng rộng rãi trong nghiên cứu cũng
  như trong những ứng dụng thương mại. Sau đây là những ứng dụng
  Smalltalk tiêu biểu đang được phát triển.
\begin{description}
\item[Giảng dạy:] EToys (Squeak), SqueakBot, BotsInc, Scratch\dots
\item[Multimedia:] Sophie, OpenCroquet, Plopp\dots
\item[Phát triển web:] Seaside, Aida, Komanche, Swazoo\dots
\item[Quản lý lưu trữ:] object oriented databases (Magma, GemStone),
  relational databases (MySQL, PostgreSQL), object relational mapping
  (Glorp).
\end{description}
}

\newcommand{\stApplicationScreenshotPicture}{olpc-etoys}
\newcommand{\stApplicationScreenshotPictureWidth}{.56}
\newcommand{\stApplicationScreenshotTerm}{Etoys và DrGeo trên máy XO-OLPC}

\newcommand{\stImageTerm}{Image (Ảnh)}

\newcommand{\stImageDefinition} {Môi trường Smalltalk có một đối tượng
  lưu trữ gọi là image, chứa mã nguồn (gồm class và method), cùng các
  đối tượng lưu trạng thái của ứng dụng. Image còn có thể chứa các
  công cụ phát triển, giúp kiểm tra và bắt lỗi chương trình.}

\newcommand{\stVMTerm}{Virtual Machine (Máy ảo)}
\newcommand{\stVMDefinition} {Máy ảo (VM) là một chương trình có khả
  năng thi hành các chương trình khác, giúp ứng dụng chạy trên nhiều
  nền tảng.}

\newcommand{\stReflexionTerm}{Reflection}
\newcommand{\stReflexionDefinition} {Reflection là một tính năng cho
  phép kiểm tra và cập nhật mã nguồn khi chương trình đang chạy.}

\newcommand{\stDynamicTypingTerm}{Dynamic typing (Định kiểu động)}
\newcommand{\stDynamicTypingDefinition} {Một số ngôn ngữ buộc lập
  trình viên định rõ kiểu của từng biến (integer, string, \dots); được
  gọi là định kiểu tĩnh (static typing). Dynamic typing không yêu cầu
  điều này, vì thế làm cho chương trình dễ tái sử dụng hơn và thay đổi
  hơn.}

\newcommand{\stBooksTerm}{Sách}
\newcommand{\stBooksDefinition}
{
\begin{itemize}
\item Nhiều sách tự do về Smalltalk:\\
  \url{http://stephane.ducasse.free.fr/FreeBooks.html}

\item Smalltalk nói chung
  \begin{itemize}
  \item \emph{Smalltalk with Style}\\
		(Edward Klimas, Suzanne Skublics and
    David A. Thomas, miễn phí))
  \item \emph{Smalltalk by Example: the Developer's Guide} --
		(Alec Sharp, miễn phí))
  \end{itemize}

\item Squeak chi tiết
  \begin{itemize}
  \item \emph{Squeak by Example} -- (2007, miễn phí)
  \item \emph{Powerful Ideas in the Classroom}\\
		(BJ Allen-Conn and Kim Rose)
  \end{itemize}
\end{itemize}
}

\newcommand{\stSmalltalkActionsTerm}{Sự kiện}
\newcommand{\stSmalltalkActionsDefinition}
{
\begin{itemize}
\item European Smalltalk User Group conferences (ESUG) Hội thảo Nhóm
  người dùng Smalltalk châu Âu). Từ năm 1993, gồm những người dùng
  Smalltalk (Smalltalkers) trong công nghiệp và trong học
  thuật.\\ \url{http://www.esug.org/conferences}
\item Hội nghị thường niên, được tổ chức ở Bắc Mỹ bởi STIC (\url{http://www.stic.st}), hội liên
  hiệp những công ty và những nhà phát triển Smalltalk.\\
  \url{http://www.smalltalksolutions.com/}
\end{itemize}
}

\newcommand{\stInternetWebsitesDefinition}
{
\begin{itemize}
\item Trang web chính thức của Squeak:\\ \url{http://www.squeak.org}
\item Wiki:\\ \url{http://wiki.squeak.org}
\item Tin tức:\\ \url{http://news.squeak.org}
\item Mailing-list cộng đồng Smalltalker Việt Nam:\\ \url{http://lists.squeakfoundation.org/mailman/listinfo/smalltalk-vn}
\end{itemize}
}

\newcommand{\stNilDefinition}{đối tượng chưa định (mặc định)}
\newcommand{\stTrueAndFalseDefinition}{đối tượng logic (boolean)}
\newcommand{\stSelfDefinition}{đối tượng hiện tại}
\newcommand{\stSuperDefinition}{đối tượng hiện tại (trong super class)}
\newcommand{\stThisContextDefinition}{stack tại thời điểm thực thi của method}
\newcommand{\stAssignmentOperatorDefinition}{gán giá trị}
\newcommand{\stReturnOperatorDefinition}{trả về kết quả của method}
\newcommand{\stTempsDeclarationOperatorDefinition}{khai báo ba biến tạm}
\newcommand{\stDollarOperatorForCharacterADefinition}{ký tự \code{a}}
\newcommand{\stLiteralArrayDefinition}{mảng gồm hai phần tử: symbol \code{\#abc} và số \code{123}}
\newcommand{\stDotOperatorDefinition}{kết thúc biểu thức}
\newcommand{\stSemiColonOperatorDefinition}{message cascade}
\newcommand{\stBlockOperatorDefinition}{block (là một đối tượng!)}

\newcommand{\stPeriodTerm}{dấu chấm}
\newcommand{\stCommentTerm}{ghi chú}
\newcommand{\stStringTerm}{xâu ký tự}

\newcommand{\stGlossaryTerm}{Thuật ngữ}

\newcommand{\stAndTerm}{và}
\newcommand{\stOrTerm}{hoặc}

\newcommand{\stBlockTerm}{Block (Khối)}
\newcommand{\stBlockDefinition}{

  Các block là các đối tượng chứa mã không được thực thi ngay. Chúng
  là cơ sở cho cấu trúc điều khiển như rẽ nhánh hay lặp. Đồng thời,
  các block có thể được dùng trong xử lý tương tác, ví dụ: gắn vào
  menu.

  \begin{displaycode}
    \#('Hello ' 'World')

    \hspace{1cm}  do: [:string | Transcript show: string].
  \end{displaycode}

  Ví dụ này gửi message \code{do:} đến một mảng các xâu với
  tham số là một block. Block sẽ được thi hành với mỗi phần từ của
  mảng một lần. Tham số block \code{string} chứa lần lượt từng phần tử
  của. Kết quả của cả biểu thức là xâu \code{'Hello '} rồi đến
  \code{'World'} được hiển thị trong Transcript một cách lần lượt.
   
}

\newcommand{\stSqueakCodeBrowserSize}{1}

%%%%%%%%%%%%%%%%%%%%%%%%%%%%%%%%%%%%%%%%%%%%%%%%%%%%%%%%%%%%%%%%%

\newcommand{\seaSubtitle}{Framework để phát triển các ứng dụng web
  công phu trong Smalltalk}

%%% Local Variables:
%%% coding: utf-8-unix
%%% mode: latex
%%% TeX-master: "../flyer"
%%% TeX-PDF-mode: t
%%% ispell-local-dictionary: "vietnamese"
%%% End:

