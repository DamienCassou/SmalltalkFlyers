\newcommand{\stSmalltalkSubtitle}
{un llenguatge de programació\\
  purament \textbf{orientat a objectes}\\
  i un entorn \textbf{dinàmic}}

\newcommand{\stSmalltalkConceptsTerm}{Conceptes Importants d'Smalltalk}
\newcommand{\stSmalltalkConceptsDefinition} {Smalltalk és un
llenguatge \emph{orientat a objectes}, amb \emph{tipat dinàmic} i una
sintaxi senzilla que es pot aprendre en \emph{quinze minuts}.
El seu principal avantatge és ser \emph{molt consistent}:

\begin{itemize}
\item tot és un objecte: classes, mètodes, nombres, etc.
\item un nombre petit de regles, sense excepcions!
\end{itemize}

Smalltalk s'executa sobre una \emph{màquina virtual}. El desenvolupament
té lloc dins d'una \emph{imatge}, en la que viuen i poden ser modificats tots els objectes.}

\newcommand{\stSmalltalkSyntaxTerm}{Sintaxi d'Smalltalk}
\newcommand{\stReservedWordsTerm}{Paraules reservades}
\newcommand{\stReservedCaractersTerm}{Caràcters reservats}

\newcommand{\stMessageSendingTerm}{Enviament de missatges}
\newcommand{\stMessageSendingDefinition}
{Un mètode és cridat enviant un missatge a un objecte, el receptor del missatge, el missatge retorna un objecte. El missatge està basat en el llenguatge natural, amb subjecte, verb i complements. Hi ha tres tipus de missatges: unari, binari i paraula clau.}

\newcommand{\stUnaryMessagesTerm}{Missatges unaris.}
\newcommand{\stUnaryMessagesDefinition}
{Un missatge unari no té arguments

\begin{displaycode}
array := Array new.

array size.
\end{displaycode}

El primer exemple crea i retorna una nova instància de la classe Array, enviant el missatge \code{new}.
El segon exemple demana la mida d'aquesta taula (\code{array}), i retorna \code{0}.}

\newcommand{\stBinaryMessagesTerm}{Missatges binaris.}
\newcommand{\stBinaryMessagesDefinition}
{Un missatge binari pren només un argument, el nom és un símbol i s'utilitza sovint
per a expressions aritmètiques.

\begin{displaycode}
3 + 4.

'Hola', ' Món'.
\end{displaycode}

El missatge \code{+} és enviat a l'objecte \code{3} amb \code{4} de paràmetre.
En el segon cas, el missatge \code{,} és enviat a la cadena  \code{'Hola'} amb \code{' Món'} de paràmetre.}

\newcommand{\stKeywordMessagesTerm}{Missatges de paraula clau.}
\newcommand{\stKeywordMessagesDefinition}
{
  Un missatge de paraula clau pot prendre un o més arguments. Els arguments s'insereixen entre cada paraula clau, després dels dos punts.
  
\begin{displaycode}
'Smalltalk' allButFirst: 5.

3 to: 10 by: 2.
\end{displaycode}

El primer exemple crida el mètode \code{allButFirst:} sobre una cadena de caràcters i
amb argument \code{5}. El mètode retorna la cadena de caràcters \code{'talk'}. 
El segon exemple retorna una col·lecció contenint els 
elements \code{3}, \code{5}, \code{7} i \code{9}.

}

\newcommand{\stDevelopmentEnvironmentTerm}{Entorn de desenvolupament}
\newcommand{\stDevelopmentEnvironmentDefinition} {La majoria de les
  implementacions d'Smalltalk proporcionen un entorn integrat
  que permet explorar el codi font i interaccionar amb objectes. Aquest entorn disposa de moltes
  eines, totes implementades en Smalltalk gràcies a la seva API de reflexió:

\begin{itemize}
\item explorador de classes i mètodes;
\item eines de \emph{refactoring};
\item inspectors d'objectes;
\item un depurador;
\item administrador i controlador de versions; 
\item i molt, molt més!
\end{itemize}

El codi pot ser inspeccionat i avaluat directament dins la imatge, amb menus i senzilles combinacions de tecles.}

\newcommand{\stImplementationTerm}{Implementacions}
\newcommand{\stImplementationDefinition}
{
Existeixen diverses implementacions disponibles:
\begin{description}
\item[Squeak:] gratuïta, \emph{open-source} i multi-plataforma.
  Desenvolupada activament per una comunitat internacional.
\item[VisualWorks:] propietària i multi-plataforma, disponible gratuïtament per a ús no comercial.
\item[Gemstone:] propietària, inclou una base de dades d'objectes molt eficient.
\item[I altres:] GNU Smalltalk, Smalltalk/X, SyX, VA Smalltalk, Dolphin\dots
\end{description}
}

\newcommand{\stSqueakCodeBrowserTerm}{L'explorador de codi de Squeak}

\newcommand{\stApplicationsTerm}{Aplicacions}
\newcommand{\stApplicationsDefinition}
{
Des de la seva creació a principis dels 80s, Smalltalk ha estat extensament utilitzat tant en recerca
com en el món comercial. Exemples actuals d'aplicacions Smalltalk que 
contribueixen a fer avançar la tecnologia del programari.
\begin{description}
\item[Docència:] EToys (Squeak), SqueakBot, BotsInc, Scratch\dots
\item[Multimèdia:] Sophie, OpenCroquet, Plopp\dots
\item[Desenvolupament web:] Seaside, Aida, Komanche, Swazoo\dots
\item[Gestió de la persistència:] bases de dades orientades a objectes (Magma, GemStone), bases de dades relacionals (MySQL, PostgreSQL), correspondència entre objectes i relacions (Glorp).
\end{description}
}

\newcommand{\stApplicationScreenshotPicture}{plopp}
\newcommand{\stApplicationScreenshotPictureWidth}{.45}
\newcommand{\stApplicationScreenshotTerm}{Una sessió de dibuix amb Plopp}

\newcommand{\stImageTerm}{Imatge}
\newcommand{\stImageDefinition}
{L'entorn Smalltalk proporciona un magatzem persistent d'objectes, la imatge. Aquesta conté el
codi de les aplicacions (classes i mètodes), objectes que mantenen l'estat de l'aplicació i fins i
tot pot incloure eines de desenvolupament per inspeccionar i depurar un programa mentre
s'executa.}

\newcommand{\stVMTerm}{Màquina Virtual}
\newcommand{\stVMDefinition}
{Una màquina virtual és un programa que és capaç d'executar altres programes. Facilita la
portabilitat d'aplicacions.}

\newcommand{\stReflexionTerm}{Reflexió}
\newcommand{\stReflexionDefinition} 
{Un llenguatge es diu que és reflexiu quan conté els mecanismes per inspeccionar i modificar el codi mentre
s'està executant el programa corresponent.}

\newcommand{\stDynamicTypingTerm}{Tipat Dinàmic}
\newcommand{\stDynamicTypingDefinition} 
{Alguns llenguatges forcen al desenvolupador a fer explícit el tipus de cada variable (integer, string\dots); això s'anomena tipat estàtic. El tipat dinàmic
no imposa aquesta restricció, i per tant fa que els programes siguin més reutilitzables i fàcils de canviar.}

\newcommand{\stBooksTerm}{Llibres}
\newcommand{\stBooksDefinition}
{
\begin{itemize}
\item Molts llibres gratuïts:\\
  \url{http://stephane.ducasse.free.fr/Books.html}

\item Smalltalk en general
  \begin{itemize}
  \item \emph{Smalltalk with Style}\\
		(Edward Klimas, Suzanne Skublics and
    David A. Thomas, gratuït)
  \item \emph{Smalltalk by Example: the Developer's Guide} --
		(Alec Sharp, gratuït)
  \end{itemize}

\item Squeak en particular
  \begin{itemize}
  \item \emph{Squeak by Example} -- (2007, gratuït)
  \item \emph{Powerful Ideas in the Classroom}\\
		(BJ Allen-Conn and Kim Rose)
  \end{itemize}
\end{itemize}
}

\newcommand{\stSmalltalkActionsTerm}{Esdeveniments}
\newcommand{\stSmalltalkActionsDefinition}
{
\begin{itemize}
\item Les conferències de l'European Smalltalk User Group (ESUG). Des de 1993,  Smalltalkers acadèmics i de sectors industrials es troben a Europa.\\
  \url{http://www.esug.org/conferences}
\item Conferència anual, organitzada a Nord-Amèrica per l'STIC (\url{http://www.stic.st}), una associació amb presència de sectors industrials i desenvolupadors d'Smalltalk.\\ 
  \url{http://www.smalltalksolutions.com/}
\end{itemize}
}

\newcommand{\stInternetWebsitesDefinition}
{
\begin{itemize}
\item Lloc web oficial d'Squeak:\\ \url{http://www.squeak.org}
\item Wiki:\\ \url{http://wiki.squeak.org}
\item Novetats:\\ \url{http://news.squeak.org}
\end{itemize}
}

\newcommand{\stNilDefinition}{objecte no definit (valor per defecte)}
\newcommand{\stTrueAndFalseDefinition}{objectes booleans}
\newcommand{\stSelfDefinition}{objecte receptor del missatge}
\newcommand{\stSuperDefinition}{objecte receptor del missatge (dins un context  de super classe)}
\newcommand{\stThisContextDefinition}{ pila d'execució del mètode actual}
\newcommand{\stAssignmentOperatorDefinition}{assignació}
\newcommand{\stReturnOperatorDefinition}{retorna un resultat des d'un mètode}
\newcommand{\stTempsDeclarationOperatorDefinition}{declaració de tres variables temporals}
\newcommand{\stDollarOperatorForCharacterADefinition}{caràcter \code{a}}
\newcommand{\stLiteralArrayDefinition}{taula (array) que conté dos literals: el símbol \code{\#abc} i el nombre \code{123}}
\newcommand{\stDotOperatorDefinition}{fi d'expressió}
\newcommand{\stSemiColonOperatorDefinition}{missatges en cascada}
\newcommand{\stBlockOperatorDefinition}{bloc de codi (és un objecte !)}

\newcommand{\stPeriodTerm}{punt}
\newcommand{\stCommentTerm}{comentari}
\newcommand{\stStringTerm}{cadena}

\newcommand{\stGlossaryTerm}{Glossari}

\newcommand{\stAndTerm}{i}
\newcommand{\stOrTerm}{o}

\newcommand{\stBlockTerm}{Bloc}
\newcommand{\stBlockDefinition}{

  Els blocs són objectes que contenen codi que no és executat immediatament.
  Són la base d'estructures de control com els condicionals o les repeticions. 
  Els blocs poden ser utilitzats per associar comportaments, ex. les
  opcions d'un menú.

  \begin{displaycode}
    \#('Hola' ' Món')

    \hspace{1cm}  do: [:string | Transcript show: string].
  \end{displaycode}

  L'exemple envia el missatge \code{do:} a una taula de cadenes de caràcters amb un bloc com a paràmetre. 
  El bloc és avaluat un cop per cada element de la taula. El paràmetre del
  bloc, \code{string}, conté cada element de la taula, un darrera l'altre.
  Com a resultat de tota l'expressió, les cadenes de caràcters
  \code{'Hola'} i després \code{' Món'} es mostren al Transcript.
}

\newcommand{\stSqueakCodeBrowserSize}{1}

%%%%%%%%%%%%%%%%%%%%%%%%%%%%%%%%%%%%%%%%%%%%%%%%%%%%%%%%%%%%%%%%%

\newcommand{\seaSubtitle}{L'entorn de treball per desenvolupar
aplicacions web sofisticades en Smalltalk}

%%% Local Variables:
%%% coding: utf-8-unix
%%% mode: latex
%%% TeX-master: "../flyer"
%%% TeX-PDF-mode: t
%%% ispell-local-dictionary: "english"
%%% End:

