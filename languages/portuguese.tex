\newcommand{\stSmalltalkSubtitle}
{uma linguagem de programação\\
  puramente \textbf{orientada a objetos}\\
  e um ambiente \textbf{dinâmico}}

\newcommand{\stSmalltalkConceptsTerm}{Conceitos importantes de Smalltalk}
\newcommand{\stSmalltalkConceptsDefinition} {Smalltalk é uma linguagem
  \emph{orientada a objetos}, \emph{dinamicamente tipada}, com uma
  sintaxe simples que pode ser aprendida em \emph{quinze minutos}.
Sua maior vantagem é ser \emph{muito consistente}:

\begin{itemize}
\item tudo é um objeto: classes, métodos, números, etc.
\item um pequeno número de regras, sem excessões!
\end{itemize}

Smalltalk roda numa \emph{máquina virtual}.  O desenvolvimento
acontece numa \emph{imagem} onde todos os objetos vivem e são
modificados.}

\newcommand{\stSmalltalkSyntaxTerm}{A sintaxe do Smalltalk}
\newcommand{\stReservedWordsTerm}{Palavras reservadas}
\newcommand{\stReservedCaractersTerm}{Caracteres reservados}

\newcommand{\stMessageSendingTerm}{Envio de mensagens}
\newcommand{\stMessageSendingDefinition}
{Um método é chamado pelo envio de uma mensagem a um objeto, o receptor da mensagem; a mensagem retorna um objeto. A mensagem é baseada em linguagem natural, tem um sujeito, um verbo e complementos.
Existem três tipos de mensagem: unária, binária e palavra-chave.}

\newcommand{\stUnaryMessagesTerm}{Mensagens unárias.}
\newcommand{\stUnaryMessagesDefinition}
{Não têm argumentos.

\begin{displaycode}
array := Array new.

array size.
\end{displaycode}

O primeiro exemplo cria e retorna uma nova instancia da classe Array
ao lhe enviar a mensagem \code{new}. O segundo exemplo pede o tamanho
desta, o que retorna \code{0}.}

\newcommand{\stBinaryMessagesTerm}{Mensagens binárias.}
\newcommand{\stBinaryMessagesDefinition}{Têm só
  um argumento, são chamadas por um símbolo e normalmente usadas para
  expressões matemáticas.

\begin{displaycode}
3 + 4.

'Hello', ' World'.
\end{displaycode}

A mensagem \code{+} é enviada para o objeto \code{3} tendo \code{4} como par\^ametro. No segundo caso a mensagem \code{,} é envidada para a cadeia de caracteres \code{'Hello'} tendo \code{' World'} como par\^ametro.}

\newcommand{\stKeywordMessagesTerm}{Mensagens de palavra-chave.}
\newcommand{\stKeywordMessagesDefinition}
{
  Podem ter um ou mais argumentos, que são inseridos entre cada palavra-chave, depois dos dois pontos.

\begin{displaycode}
'Smalltalk' allButFirst: 5.

3 to: 10 by: 2.
\end{displaycode}

O primeiro exemplo chama o método \code{allButFirst:} numa cadeia de caracteres e
passa o argumento \code{5}. O método retorna a cadeia de caracteres
\code{'talk'}. O segundo exemplo retorna uma coleção contendo
os elementos \code{3}, \code{5}, \code{7} e \code{9}.
}

\newcommand{\stDevelopmentEnvironmentTerm}{Ambiente de desenvolvimento}
\newcommand{\stDevelopmentEnvironmentDefinition} {A maioria de suas
  implementações vem com um ambiente de desenvolvimento
  integrado que permite navegar no código e interajir com os objetos.
  Muitas ferramentas estão disponíveis, todas
  implementadas em Smalltalk graças à API de reflexão:

\begin{itemize}
\item um navegador de classe e método;
\item ferramentas de refatoração;
\item inspetores de objetos;
\item um depurador;
\item ferramentas de controle de releases e controle de versão;
\item e muito, muito mais!
\end{itemize}

O código pode ser inspecionado e avaliado diretamente na imagem, usando simples combinações de teclas e menus.  }

\newcommand{\stImplementationTerm}{Implementações}
\newcommand{\stImplementationDefinition}
{
Existem várias implementações de Smalltalk disponíveis:
\begin{description}
\item[Squeak:] gratuita, de código aberto e
  multi-plataforma. Desenvolvida ativamente por uma comunidade internacional.
\item[Visual Works:] proprietaria e multi-plataforma, disponível gratuitamente
  para uso não-comercial.
\item[Gemstone:] implementação proprietária que inclui um banco de dados
  altamente eficiente.
\item[And others:] GNU Smalltalk, Smalltalk/X, SyX, VA Smalltalk, Dolphin\dots
\end{description}
}

\newcommand{\stSqueakCodeBrowserTerm}{O navegador de código do Squeak}

\newcommand{\stApplicationsTerm}{Aplicações}
\newcommand{\stApplicationsDefinition}
{
Desde sua criação no início dos anos 80, Smalltalk tem sido amplamente usado tanto em 
pesquisa acadêmica com em aplicações comerciais. Aqui estão algumas aplicações em
Smalltalk atuais que estão evoluindo a tecnologia do software.
\begin{description}
\item[Ensino:] EToys (Squeak), SqueakBot, BotsInc, Scratch\dots
\item[Multimídia:] Sophie, OpenCroquet, Plopp\dots
\item[Desenvolvimento web:] Seaside, Aida, Komanche, Swazoo\dots
\item[Persistência:] bancos de dados orientados a objeto (Magma, GemStone), relacionais (MySQL, PostgreSQL) e mapeamentos objeto-relacional (Glorp).
\end{description}
}

\newcommand{\stPloppDrawingSessionTerm}{Uma sesso de desenho no Plopp}

\newcommand{\stImageTerm}{Imagem}
\newcommand{\stImageDefinition}
{O ambiente do Smalltalk contém um repositório de objetos persistente, a
  imagem. Esta contém código de aplicações (classes and métodos), objetos
  contendo o estado das aplicaçães e até inclui as ferramentas de
  desenvolvimento para inspecionar e depurar o programa enquantoeventos ele está executando.}

\newcommand{\stVMTerm}{Máquina virtual}
\newcommand{\stVMDefinition}
{Uma máquina virtual é um programa que é capaz de executar outros
  programas. Ele facilita a portabilidade de aplicações.}

\newcommand{\stReflexionTerm}{Reflexão computacional}
\newcommand{\stReflexionDefinition} {Um linguagem é dita reflexiva quando
  contem mecanismos para inspecionar e modificar o código durante a execução
  do programa.}

\newcommand{\stDynamicTypingTerm}{Tipagem dinâmica}
\newcommand{\stDynamicTypingDefinition} {Algumas linguagens forçam
  o desenvolvedor a indicar o tipo de cada variável (inteiro,
  cadeia de caracteres\dots); isto é chamado tipagem estática. Tipagem dinâmica
 não impõe está restrição, o que torna os programas mais reusáveis e fáceis de manter.}

\newcommand{\stBooksTerm}{Livros}
\newcommand{\stBooksDefinition}
{
\begin{itemize}
\item Vários livros gratuítos:\\
  \url{http://stephane.ducasse.free.fr/Books.html}

\item Smalltalk em geral
  \begin{itemize}
  \item \emph{Smalltalk with Style}\\
		(Edward Klimas, Suzanne Skublics and
    David A. Thomas, free)
  \item \emph{Smalltalk by Example: the Developer's Guide} --
		(Alec Sharp, free)
  \end{itemize}

\item Squeak em particular
  \begin{itemize}
  \item \emph{Squeak by Example} -- (2007, free)
  \item \emph{Powerful Ideas in the Classroom}\\
		(BJ Allen-Conn and Kim Rose)
  \end{itemize}
\end{itemize}
}

\newcommand{\stSmalltalkActionsTerm}{Eventos}
\newcommand{\stSmalltalkActionsDefinition}
{
\begin{itemize}
\item Confer\^encias do Grupo de Usuários Europeu de Smalltalk (ESUG). Desde 1993, usuários de Smalltalk da industria e da academia se encontram em um país europeu.\\
  \url{http://www.esug.org/conferences}
\item Conferência anual, organisada na America do Norte pelo STIC (\url{http://www.stic.st}), uma associação com atores industriais e desenvolvedores de Smalltalk.\\
  \url{http://www.smalltalksolutions.com/}
\end{itemize}
}

\newcommand{\stInternetWebsitesDefinition}
{
\begin{itemize}
\item Official Squeak website:\\ \url{http://www.squeak.org}
\item Wiki:\\ \url{http://wiki.squeak.org}
\item News:\\ \url{http://news.squeak.org}
\end{itemize}
}

\newcommand{\stNilDefinition}{objeto indefinido (valor inicial das variáveis)}
\newcommand{\stTrueAndFalseDefinition}{objetos boleanos}
\newcommand{\stSelfDefinition}{objeto atual}
\newcommand{\stSuperDefinition}{objeto atual no contexto da superclasse}
\newcommand{\stThisContextDefinition}{pilha de execução do método atual}
\newcommand{\stAssignmentOperatorDefinition}{atribuição}
\newcommand{\stReturnOperatorDefinition}{retorna um resultado dum método}
\newcommand{\stTempsDeclarationOperatorDefinition}{declaração de três variáveis temporárias}
\newcommand{\stDollarOperatorForCharacterADefinition}{charactere \code{a}}
\newcommand{\stLiteralArrayDefinition}{array contendo dois literais: o símbolo \code{\#abc} e o número \code{123}}
\newcommand{\stDotOperatorDefinition}{fim da expressão}
\newcommand{\stSemiColonOperatorDefinition}{messagem em cascata}
\newcommand{\stBlockOperatorDefinition}{bloco de código (é um objeto !)}

\newcommand{\stPeriodTerm}{ponto}
\newcommand{\stCommentTerm}{comentário}
\newcommand{\stStringTerm}{cadeia de caracteres}

\newcommand{\stGlossaryTerm}{Glossário}

\newcommand{\stAndTerm}{e}
\newcommand{\stOrTerm}{ou}

\newcommand{\stBlockTerm}{Bloco}
\newcommand{\stBlockDefinition}{

  Blocos são objetos contendo código que não é executado imediatamente.
  Eles são a base das estruturas de controle como condicionais ou
  laços. Inclusive blocos podem ser usados para acrescentar comportamento, p. ex.,
  a itens de menu.

  \begin{displaycode}
    \#('Hello ' 'World')

    \hspace{1cm}  do: [:string | Transcript show: string].
  \end{displaycode}

  O exemplo envia a mensagem \code{do:} para uma array de cadeia de caracteres com
  um bloco como par\^ametro. O bloco é avaliado uma vez para cada elemento
  da array. O par\^ametro do bloco, \code{string}, contém cada
  elemento da array um após o outro. Como resultado da expressão
  inteira as cadeias de caracteres \code{'Hello '} e então \code{'World'} são
  mostradas.
   
}

\newcommand{\stSqueakCodeBrowserSize}{1}

%%%%%%%%%%%%%%%%%%%%%%%%%%%%%%%%%%%%%%%%%%%%%%%%%%%%%%%%%%%%%%%%%

\newcommand{\seaSubtitle}{O framework para desenvolvimento de aplicações web
  sofisticadas em Smalltalk}

%%% Local Variables:
%%% coding: utf-8-unix
%%% mode: latex
%%% TeX-master: "../flyer"
%%% TeX-PDF-mode: t
%%% ispell-local-dictionary: "english"
%%% End:

