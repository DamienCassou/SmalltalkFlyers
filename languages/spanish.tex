\newcommand{\stSmalltalkSubtitle}
{un lenguaje de programación\\
  puramente \textbf{orientado a objetos}\\
  y un ambiente \textbf{dinámico}}

\newcommand{\stSmalltalkConceptsTerm}{Conceptos importantes de Smalltalk}
\newcommand{\stSmalltalkConceptsDefinition} {Smalltalk es un
lenguaje \emph{orientado a objetos} y \emph{dinámicamente tipado}, con una
sintáxis simple que puede ser aprendida en \emph{quince minutos}.
Su mayor ventaja proviene del hecho que es \emph{muy consistente}:

\begin{itemize}
\item todo es un objeto: clases, métodos, números, etc.
\item una pequeña cantidad de reglas, y ninguna excepción!
\end{itemize}

Smalltalk corre sobre una \emph{máquina virtual}.  El desarrollo
se realiza en una \emph{imagen} en donde todos los objetos viven 
y son modificados.}

\newcommand{\stSmalltalkSyntaxTerm}{Sintaxis de Smalltalk}
\newcommand{\stReservedWordsTerm}{Palabras reservadas}
\newcommand{\stReservedCaractersTerm}{Caracteres reservados}

\newcommand{\stMessageSendingTerm}{Envío de mensajes}
\newcommand{\stMessageSendingDefinition}
{Un método es invocado al enviar un mensaje a un objeto (el receptor del mensaje) y el mensaje devuelve un objeto. El mensaje esta basado en el lenguaje natural, conformado por un sujeto, un verbo y argumentos.
Existen tres tipos de mensajes: unario, binario y de palabra clave.}

\newcommand{\stUnaryMessagesTerm}{Mensajes unarios.}
\newcommand{\stUnaryMessagesDefinition}
{El mensaje unario no tiene argumentos.

\begin{displaycode}
array := Array new.

array size.
\end{displaycode}

El primer ejemplo crea y devuelve una nueva instancia de la clase 
Array, al enviarle el mensaje \code{new}. El segundo ejemplo solicita
el tamaño de ese arreglo y devuelve \code{0}.}

\newcommand{\stBinaryMessagesTerm}{Mensajes binarios.}
\newcommand{\stBinaryMessagesDefinition}{El mensaje binario tiene solo
  un argumento, su nombre es un símbolo y es usado frecuentemente para 
  expresiones aritméticas.

\begin{displaycode}
3 + 4.

'Hola', ' Mundo'.
\end{displaycode}

El mensaje \code{+} es enviado al objeto \code{3} con el objeto \code{4} como parámetro. En el segundo caso, el mensaje \code{,} es enviado a la cadena de caracteres \code{'Hola'} con \code{' Mundo'} como parámetro.}

\newcommand{\stKeywordMessagesTerm}{Mensajes de palabra clave.}
\newcommand{\stKeywordMessagesDefinition}
{
  Un mensaje de palabra clave puede tener uno o más argumentos. Los argumentos se sitúan entre cada palabra clave, después de los dos puntos.

\begin{displaycode}
'Smalltalk' allButFirst: 5.

3 to: 10 by: 2.
\end{displaycode}

El primer ejemplo invoca el método \code{allButFirst:} sobre una cadena de caracteres y
con el argumento \code{5}. El método devuelve la cadena de caracteres
\code{'talk'}. El segundo ejemplo devuelve una colección con los 
elementos \code{3}, \code{5}, \code{7} y \code{9}.
}

\newcommand{\stDevelopmentEnvironmentTerm}{Entorno de desarrollo}
\newcommand{\stDevelopmentEnvironmentDefinition} {La mayoría de
  las implementaciones de Smalltalk proveen un entorno de desarrollo 
  integrado que permite explorar el código fuente e interactuar 
  con los objetos. Muchas herramientas estan disponibles, todas
  implementadas en Smalltalk gracias a su reflexiva API:

\begin{itemize}
\item browser de clases y de métodos;
\item herramientas de refactoring;
\item inspectores de objetos;
\item un depurador;
\item herramientas de administración de releases y de control de versiones;
\item y mucho, mucho más!
\end{itemize}

El código puede ser inspeccionado y ejecutado directamente en la imagen, usando combinaciones de teclas simples y opciones de menús.}

\newcommand{\stImplementationTerm}{Implementaciones}
\newcommand{\stImplementationDefinition}
{
Existen varias implementaciones de Smalltalk disponibles:
\begin{description}
\item[Squeak:] implementación gratuita, open-source y 
  multi-plataforma. Desarrollada activamente por una comunidad internacional.
\item[Visual Works:] implementación proprietaria y multi-plataforma, disponible 
  gratuitamente para uso no comercial.
\item[Gemstone:] implementación propietaria que incluye una base de objetos 
  de alto rendimiento
\item[Y otros:] GNU Smalltalk, Smalltalk/X, SyX, VA Smalltalk, Dolphin\dots
\end{description}
}

\newcommand{\stSqueakCodeBrowserTerm}{El browser de clases de Squeak}

\newcommand{\stApplicationsTerm}{Aplicaciones}
\newcommand{\stApplicationsDefinition}
{
Desde su creación en los principios de los 80's, Smalltalk ha sido extensamente usado en
investigación académica así como en aplicaciones comerciales.  Aquí hay algunas aplicaciones Smalltalk
actuales que contribuyen al avance de la tecnología del software.
\begin{description}
\item[Educación:] EToys (Squeak), SqueakBot, BotsInc, Scratch\dots
\item[Multimedia:] Sophie, OpenCroquet, Plopp\dots
\item[Desarrollo web:] Seaside, Aida, Komanche, Swazoo\dots
\item[Persistencia:] bases de objetos (Magma, GemStone), bases de datos relacionales (MySQL, PostgreSQL), mapeo objeto-relacional (Glorp).
\end{description}
}

\newcommand{\stPloppDrawingSessionTerm}{Una sesión de dibujos de Plopp}

\newcommand{\stImageTerm}{Imagen}
\newcommand{\stImageDefinition}
{El ambiente de Smalltalk provee un almacenamiento persistente de objetos, la
  imagen. Esta contiene código de las aplicaciones (clases y métodos), objetos
  que mantienen el estado de las aplicaciones y hasta puede incluir las herramientas de 
  desarrollo para inspeccionar y depurar el programa mientras se esta ejecutando.}

\newcommand{\stVMTerm}{Máquina Virtual}
\newcommand{\stVMDefinition}
{Una máquina virtual es un programa que es capaz de ejecutar otros
  programas. Facilita la portabilidad de la aplicación.}

\newcommand{\stReflexionTerm}{Reflexión}
\newcommand{\stReflexionDefinition} {Un lenguaje es reflexivo cuando
  provee mecanismos para inspeccionar y modificar el código de un programa durante 
  la ejecución del mismo.}

\newcommand{\stDynamicTypingTerm}{Tipeo dinámico}
\newcommand{\stDynamicTypingDefinition} {Algunos lenguajes fuerzan
  al desarrollador a especificar el tipo de cada variable (integer,
  string\dots); esto es llamado tipeo estático. Tipeo dinámico
no impone esta restricción, por lo que contribuye a una mayor reusabilidad y a que los programas sean más fáciles de modificar.}

\newcommand{\stBooksTerm}{Libros}
\newcommand{\stBooksDefinition}
{
\begin{itemize}
\item Numerosos libros gratuitos:\\
  \url{http://stephane.ducasse.free.fr/Books.html}

\item Smalltalk en general
  \begin{itemize}
  \item \emph{Smalltalk with Style}\\
		(Edward Klimas, Suzanne Skublics and
    David A. Thomas, gratiuto)
  \item \emph{Smalltalk by Example: the Developer's Guide} --
		(Alec Sharp, gratuito)
  \end{itemize}

\item Squeak en particular
  \begin{itemize}
  \item \emph{Squeak by Example} -- (2007, gratiuto)
  \item \emph{Powerful Ideas in the Classroom}\\
		(BJ Allen-Conn and Kim Rose)
  \item \emph{Programando con Smalltalk} -- (Diego Gómez Deck)
  \end{itemize}
\end{itemize}
}

\newcommand{\stSmalltalkActionsTerm}{Eventos}
\newcommand{\stSmalltalkActionsDefinition}
{
\begin{itemize}
\item Conferencias de la European Smalltalk User Group (ESUG). Desde 1993, Smalltalkers de sectores industriales y académicos se reunen en un país europeo.\\
  \url{http://www.esug.org/conferences}
\item Conferencia anual, organizada en América del Norte por la STIC (\url{http://www.stic.st}), una asociación compuesta por sectores industriales y desarrolladores de Smalltalk.\\
  \url{http://www.smalltalksolutions.com/}
\end{itemize}
}

\newcommand{\stInternetWebsitesDefinition}
{
\begin{itemize}
\item Sitio oficial de Squeak:\\ \url{http://www.squeak.org}
\item Wiki:\\ \url{http://wiki.squeak.org}
\item Novedades:\\ \url{http://news.squeak.org}
\end{itemize}
}

\newcommand{\stNilDefinition}{objeto indefinido (valor por defecto de las variables)}
\newcommand{\stTrueAndFalseDefinition}{objetos booleanos}
\newcommand{\stSelfDefinition}{objeto receptor del mensaje}
\newcommand{\stSuperDefinition}{objeto receptor del mensaje en el contexto de la super clase}
\newcommand{\stThisContextDefinition}{objeto contexto de ejecución del método actual}
\newcommand{\stAssignmentOperatorDefinition}{asignación}
\newcommand{\stReturnOperatorDefinition}{devuelve el resultado de un método}
\newcommand{\stTempsDeclarationOperatorDefinition}{declaración de variables temporales}
\newcommand{\stDollarOperatorForCharacterADefinition}{caracter \code{a}}
\newcommand{\stLiteralArrayDefinition}{arreglo con literales: el símbolo \code{\#abc} y el número \code{123}}
\newcommand{\stDotOperatorDefinition}{fin de expresión}
\newcommand{\stSemiColonOperatorDefinition}{mensajes en cascada}
\newcommand{\stBlockOperatorDefinition}{bloque de código (es un objeto !)}

\newcommand{\stPeriodTerm}{punto}
\newcommand{\stCommentTerm}{comentario}
\newcommand{\stStringTerm}{cadena de caracteres}

\newcommand{\stGlossaryTerm}{Glosario}

\newcommand{\stAndTerm}{y}
\newcommand{\stOrTerm}{o}


\newcommand{\stBlockTerm}{Bloques  - MISSING TRANSLATION -}
\newcommand{\stBlockDefinition}{

  Blocks are objects containing code that is not executed immediately.
  They are the basis for control structures like conditionals or
  loops.  Also, blocks can be used to attach behavior, e.g., to menu
  items.
  
%   Los bloques se utilizan para delimitar una secuencia parametrizada
%   de instrucciones que se evaluarán diferidamente. También se los
%   puede ver como funciones anónimas con o sin argumentos.


  \begin{displaycode}
    \#('Holla ' 'Mundo')

    \hspace{1cm}  do: [:string | Transcript show: string].
  \end{displaycode}

  El ejemplo envía el mensaje \code{do:} a un array de strings con un
  bloque como parámetro. El bloque se evalúa una vez por cada elemento
  del array. El parámetro del bloque, \code{string}, contiene cada
  elemento del array, uno a la vez. Como resultado de toda la
  expresión se muestran en el trascript los strings \code{'Hola '} y
  luego \code{'Mundo'}.

}

\newcommand{\stSqueakCodeBrowserSize}{0.1}

%%% Local Variables:
%%% coding: utf-8-unix
%%% mode: latex
%%% TeX-master: flyer
%%% TeX-PDF-mode: t
%%% ispell-local-dictionary: "spanish"
%%% End:

