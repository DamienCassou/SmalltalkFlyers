\newcommand{\stSmalltalkSubtitle}
{a programming language\\
  purely \textbf{object oriented}\\
  and a \textbf{dynamic} environment}

\newcommand{\stSmalltalkConceptsTerm}{Smalltalk important concepts}
\newcommand{\stSmalltalkConceptsDefinition} {Smalltalk is an
  \emph{object oriented} language, \emph{dynamically types}, with a
minimal syntax which can be learnt in \emph{fifteen minutes}.
It's main advantage comes from the fact that it is \emph{very coherent}:

\begin{itemize}
\item everything is an object: classes, methods, numbers, etc.
\item a really small amount of rules and no exception.
\end{itemize}

Smalltalk runs thanks to a \emph{virtual machine}.  The development
happens in an \emph{image} in which all objects live and can be
modified.}

\newcommand{\stSmalltalkSyntaxTerm}{Smalltalk syntax}
\newcommand{\stReservedWordsTerm}{Reserved words}
\newcommand{\stReservedCaractersTerm}{Reserved characters}

\newcommand{\stMessageSendingTerm}{Message sending}
\newcommand{\stMessageSendingDefinition}
{A method call is done with a message send. The message is based on a natural language, with a subject, a verb and complements.
All message send return an object. All messages are sent to an objects, called the message receiver. There exist three message types: unary, binray and keyword.}

\newcommand{\stUnaryMessagesTerm}{Unary messages.}
\newcommand{\stUnaryMessagesDefinition}
{A unary message does not have any argument.

\begin{displaycode}
array := Array new.

array size.             ==> 0
\end{displaycode}

The first example creates and returns a new instance of the Array
class, sending it the message \code{new}. The second example asks for
the size of the array which returns \code{0}.}

\newcommand{\stBinaryMessagesTerm}{Binary messages.}
\newcommand{\stBinaryMessagesDefinition}{A binary message takes only
  one argument, is named by a symbol and is often used for arithmetic
  expressions.

\begin{displaycode}
3 + 4.                ==> 7

'Hello', ' World'.    ==> 'Hello World'
\end{displaycode}

The \code{+} message is sent to the object \code{3} with \code{4} as a parameter. In the second case, the message \code{,} is sent to the string \code{'Hello'} with \code{' World'} as a parameter.}

\newcommand{\stKeywordMessagesTerm}{Keyword messages.}
\newcommand{\stKeywordMessagesDefinition}
{
  A keyword message can take one or more arguments. The arguments are inserted between each keyword, after each semicolon.

\begin{displaycode}
'Smalltalk' allButFirst: 5.    ==> 'talk'

3 to: 10 by: 2. ==> Interval with 3,5,7 and 9
\end{displaycode}

The first example call the method \code{allButFirst:} on a string and
pass the argument \code{5}. The method returns the string
\code{'talk'}. The second example returns a collection containing
elements \code{3}, \code{5}, \code{7} and \code{9}.
}

\newcommand{\stDevelopmentEnvironmentTerm}{Development environment}
\newcommand{\stDevelopmentEnvironmentDefinition} {Most of the
  Smalltalk implementations comes with an integrated development
  environment which allows for browsing the source code and
  interacting with the objects. A lots of tools are available, all
  implemented in Smalltalk thanks to a reflection API:

\begin{itemize}
\item a class and method browser;
\item refactoring tools;
\item object inspectors;
\item a debugger;
\item etc.
\end{itemize}

The environment permits the code evaluation by a simple keystroke,
immediately viewing the result.  }

\newcommand{\stImplementationTerm}{Implementations}
\newcommand{\stImplementationDefinition}
{
There exist different Smalltalk implementations:
\begin{description}
\item[Squeak] : free, open-source and multi-platform
  implementation. Actively developed by an international community.
\item[Visual Works] : proprietary and multi-platform, freely available
  for a non commercial use.
\item[Gemstone] : proprietary implementation which includes an highly
  efficient object database.
\item[And others] : GNU Smalltalk, Smalltalk/X, SyX, VA Smalltalk, Dolphin\dots
\end{description}
}

\newcommand{\stSqueakCodeBrowserTerm}{The Squeak code browser}

\newcommand{\stApplicationsTerm}{Applications}
\newcommand{\stApplicationsDefinition}
{
\begin{description}
\item[Teaching] : EToys (Squeak), SqueakBot, BotsInc, Scratch\dots
\item[Multimedia] : Sophie, OpenCroquet, Plopp\dots
\item[Web development] : Seaside, Aida, Komanche, Swazoo\dots
\item[Persistence management] : object oriented databases (Magma, GemStone), relational databases (MySQL, PostgreSQL), object relational mapping (Glorp).
\end{description}
}

\newcommand{\stPloppDrawingSessionTerm}{A Plopp drawing session}

\newcommand{\stImageTerm}{Image}
\newcommand{\stImageDefinition}
{The Smalltalk environment contains a persistent object store, the
  image. It contains application code (classes and methods), objects
  constituting application state and can even include the development
  environment to inspect and debug the program while it is executing.}

\newcommand{\stVMTerm}{Virtual Machine}
\newcommand{\stVMDefinition}
{A virtual machine is a program which is capable of executing other
  programs. It eases application portability.}

\newcommand{\stReflexionTerm}{Reflexion}
\newcommand{\stReflexionDefinition} {A language is said reflexive when
  it contains mechanisms to inspect and modify code during a program
  execution.}

\newcommand{\stDynamicTypingTerm}{Dynamic typing}
\newcommand{\stDynamicTypingDefinition} {Some languages are forcing
  the developer to indicate the type of each variable (integer,
  string\dots). This is called static typing. With dynamic typing, the
  developer does not limit its variables to a given type.}

\newcommand{\stBooksTerm}{Books}
\newcommand{\stBooksDefinition}
{
\begin{itemize}
\item Numerous free books:\\
  \url{http://stephane.ducasse.free.fr/Books.html}

\item Smalltalk in general
  \begin{itemize}
  \item \emph{Smalltalk with Style}\\
		(Edward Klimas, Suzanne Skublics and
    David A. Thomas, free)
  \item \emph{Smalltalk by Example: the Developer's Guide} --
		(Alec Sharp, free)
  \end{itemize}

\item Squeak in particular
  \begin{itemize}
  \item \emph{Squeak by Example} -- (2007, free)
  \item \emph{Powerful Ideas in the Classroom}\\
		(BJ Allen-Conn and Kim Rose)
  \end{itemize}
\end{itemize}
}

\newcommand{\stSmalltalkActionsTerm}{Actions}
\newcommand{\stSmalltalkActionsDefinition}
{
\begin{itemize}
\item European Smalltalk User Group conferences (ESUG). Since 1993, industrial and academic Smalltalkers meet in an European country.\\
  \url{http://www.esug.org/conferences}
\item Annual conference, organised in North America by the STIC, an association with industrial actors and Smalltalk editors.\\
  \url{http://www.smalltalksolutions.com/}
\end{itemize}
}

\newcommand{\stInternetWebsitesDefinition}
{
\begin{itemize}
\item Official Squeak website:\\ \url{http://www.squeak.org}
\item Wiki:\\ \url{http://wiki.squeak.org}
\end{itemize}
}

\newcommand{\stNilDefinition}{undefined object (default variable values).}
\newcommand{\stTrueAndFalseDefinition}{boolean objects.}
\newcommand{\stSelfDefinition}{current object.}
\newcommand{\stSuperDefinition}{current object in the superclass context.}
\newcommand{\stThisContextDefinition}{run-time stack of the current method.}
\newcommand{\stAssignmentOperatorDefinition}{assignment.}
\newcommand{\stReturnOperatorDefinition}{return a result from a method.}
\newcommand{\stTempsDeclarationOperatorDefinition}{declaration of three  temporary variables.}
\newcommand{\stDollarOperatorForCharacterADefinition}{character \code{a}.}
\newcommand{\stLiteralArrayDefinition}{array containing two literals \code{\#abc} and \code{123}.}
\newcommand{\stDotOperatorDefinition}{terminate expressions.}
\newcommand{\stSemiColonOperatorDefinition}{message cascade.}
\newcommand{\stBlockOperatorDefinition}{code block (it's an object !).}

\newcommand{\stCommentTerm}{comment}
\newcommand{\stStringTerm}{string}

\newcommand{\stGlossaryTerm}{Glossary}

\newcommand{\stAndTerm}{and}
\newcommand{\stOrTerm}{or}


%%% Local Variables:
%%% coding: utf-8-unix
%%% mode: latex
%%% TeX-master: flyer
%%% TeX-PDF-mode: t
%%% ispell-local-dictionary: "english"
%%% End:
